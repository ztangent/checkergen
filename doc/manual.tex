\documentclass[12pt,titlepage]{article}

%Preamble
\usepackage{a4wide}
%\usepackage{savetrees}
\usepackage{setspace}
%\usepackage{graphicx}
%\usepackage{amsmath}
%\usepackage{float}
\usepackage{listings}
\usepackage{color}
%Use Helvetica clone as font
\usepackage[T1]{fontenc}
\usepackage{tgheros}
\usepackage[colorlinks=true,linkcolor=black]{hyperref}

%Make sans serif
\renewcommand*\familydefault{\sfdefault}
%Change default monotype font
\renewcommand*\ttdefault{txtt}
\definecolor{light-gray}{gray}{0.95}

%Set listings language to bash
\lstset{
  language=bash,
  basicstyle=\ttfamily,
  keywordstyle=,
  backgroundcolor=\color{light-gray}}

\newenvironment{compactemize}
{\begin{itemize}
  \setlength{\itemsep}{1.5pt}
  \setlength{\parskip}{0pt}
  \setlength{\parsep}{0pt}}
{\end{itemize}}

\begin{document}

\title{\textbf{checkergen user manual}}
\author{}
\date{published \today}

\maketitle

\tableofcontents
\pagebreak

\onehalfspacing

\section{Introduction}

Checkergen is a simple program written in Python which generates
flashing checkerboard patterns for use in psychophysics experiments. A
command line interface (CLI) is provided for the setting of
parameters, and output either to screen or to file is provided.

In checkergen, you can create checkerboard patterns and vary the
size, position, colors, amount of distortion along the x and y axes,
as well as the frequency and phase of flashing.

You can then display these patterns on screen, and at the same time
send trigger signals through a serial or parallel port to a data
acquisition machine. Alternatively, you can export the patterns as a
series of image files which can then be displayed by another
stimulus-presenting program.

\section{Requirements}

\begin{compactemize}
\item Python 2.7 or Python 2.6 + argparse module
\item pyglet (version >= 1.1.4 recommended)
\item pySerial if you want to send trigger signals via a serial port
\item pyParallel if you want to send trigger signals via a parallel port
\begin{compactemize}
\item giveio driver required on Windows
\item see pyParallel homepage for other requirements
\end{compactemize}
\item pywin32 if you want to increase process priority on Windows
\item OpenGL installation with the \emph{GL\_EXT\_framebuffer\_object}
  extension for export and scale-to-fullscreen functionality
\end{compactemize}

\section{Running \emph{checkergen}}

Open a command prompt or terminal and make sure you are in the same
directory as the file \texttt{checkergen.py}. Also make sure that Python
is on your system path. On a UNIX-like OS, enter:
\begin{lstlisting}
./checkergen.py
\end{lstlisting}
On Windows, enter:
\begin{lstlisting}
checkergen.py
\end{lstlisting}

Once you run checkergen, the checkergen prompt will appear:
\begin{lstlisting}
Enter `help' for a list of commands.
Enter `quit' or Ctrl-D to exit.
(ckg) 
\end{lstlisting}

Checkergen can be supplied with the path to a project file as an
argument, such that that project will be loaded upon startup. For
other options, run checkergen with \lstinline{--help} as an option.

\section{Projects}

Stimuli you create in checkergen are stored as
\emph{projects}. Checkergen projects contain all your checkerboards,
as well as various other settings like framerate in frames per second
(fps) and background color. Projects are saved as XML files with 
\texttt{.ckg} extension. The name of the project is always the same as
the name of the file it is saved as, excluding the extension.

\subsection{Creating a new project}

To create a new project, enter: 
\begin{lstlisting}
(ckg) new [projectname]
\end{lstlisting}
\lstinline{projectname} is optional and defaults to `untitled'. All
project settings are initialized to reasonable defaults.

\subsection{Saving a project}
To save the project you've just created, enter: 
\begin{lstlisting}
(ckg) save [path/to/file.ckg]
\end{lstlisting}
If the path is not specified, your project will be saved to the
current working directory as \lstinline{projectname.ckg}. The path can
either be absolute or relative, but must always refer to a file with a
\texttt{.ckg} extension. The name of the saved file becomes the project
name.

\subsection{Opening an existing project}
To open an existing project, enter:
\begin{lstlisting}
(ckg) open path/to/file.ckg
\end{lstlisting}
As with \texttt{save}, the specified file must have the \texttt{.ckg}
extension. The current working directory will be changed to the
directory containing the specified file.

\subsection{Closing the project}
To close the project, enter:
\begin{lstlisting}
(ckg) close
\end{lstlisting}
If unsaved changes exist in the project, you will be prompted to save
the file first.

\subsection{Project settings}

Project settings can be edited using the \texttt{set} command. The
usage is as follows:
\begin{lstlisting}
set [--name NAME] [--fps FPS] [--res WIDTH,HEIGHT]
    [--bg COLOR] [--pre SECONDS] [--post SECONDS] 
    [--cross_cols COLOR1,COLOR2] [--cross_times TIME1,TIME2]
\end{lstlisting}
A full description can be seen by entering \lstinline{help set} into
the checkergen prompt.

An important point to note is to always remember to \texttt{set} the
fps of your project to the framerate of the monitor you are using, as
there is no auto-detect functionality. Otherwise, the pattern
animation will not be displayed at the correct speed. For example, if
your monitor has a 120 Hz refresh rate, do:
\begin{lstlisting}
(ckg) set --fps 120
\end{lstlisting}

\subsection{Listing project information}

To get an overview of the current project, enter the \texttt{ls}
command. This will list all project settings, as well as all display
groups, checkerboards, and their properties. The following is some
sample output:

{
\small
\begin{lstlisting}
(ckg) ls
***************************PROJECT SETTINGS***************************
            name     fps     resolution           bg color
         2groups     120        800,600        127,127,127
               pre-display               post-display
                         0                          0
              cross colors                cross times
             0;0;0,255;0;0                 Infinity,1

*******************************GROUP 0********************************
         pre-display              display         post-display
                   5                    5                    0
   shape id       dims      init_unit       end_unit       position
          0        5,5          40,40          50,50        380,320
          1        5,5          40,40          50,50        420,320

   shape id                      colors       anchor   freq   phase
          0           0;0;0,255;255;255  bottomright      2       0
          1           0;0;0,255;255;255   bottomleft      1      90

*******************************GROUP 1********************************
         pre-display              display         post-display
                   5                    5                    0
   shape id       dims      init_unit       end_unit       position
          0        5,5          40,40          50,50        380,320
          1        5,5          40,40          50,50        420,320

   shape id                      colors       anchor   freq   phase
          0           0;0;0,255;255;255  bottomright      1     180
          1           0;0;0,255;255;255   bottomleft      2     270
\end{lstlisting}
}

\section{Display groups}

In order to display different sets of checkerboards at different
points in time, checkerboards are organized into \emph{display}
\emph{groups}. All checkerboards in a display group are displayed
together for a certain duration, after which the next display group
takes over.

In addition to specifying the duration of a display group, you can
also specify the duration a blank screen is shown before and
after the display group is actually displayed.

\subsection{Creating a display group}
Display groups are created using \texttt{mkgrp}:
\begin{lstlisting}
(ckg) mkgrp [pre] [disp] [post]
\end{lstlisting}
where
\begin{compactemize}
\item \texttt{pre} is the time in seconds a blank screen is shown
  before the checkerboards in a display group are displayed,
\item \texttt{disp} is the time in seconds the checkerboards are
  actually displayed on-screen,
\item \texttt{post} is the time in seconds a blank screen is shown
  after the checkerboards in a display group are displayed.
\end{compactemize}
\texttt{pre} and \texttt{post} default to \texttt{0}, while
\texttt{disp} defaults to \texttt{Infinity}. Creating a display group
automatically makes it the current active context for creation and
editing of checkerboards.

\subsection{Changing the active group}
All commands that manipulate checkerboards are directed to the display
group that is currently active for editing. To change the current
active display group, do:
\begin{lstlisting}
(ckg) chgrp [group_id]
\end{lstlisting}
where \lstinline{group_id} is the ID number of the group you want to
switch to. The first group you create will have ID number 0, the next
will have ID number 1, and so on. If \lstinline{group_id} is not
specified, then checkergen prints a message telling you which is the
current active display group.

\subsection{Editing display groups}
\texttt{edgrp} can be used to edit display groups that have already
been created. It is used as follows:
\begin{lstlisting}
edgrp [--pre SECONDS] [--disp SECONDS] [--post SECONDS]
      list_of_group_ids
\end{lstlisting}
The options you can specify are the same as in \texttt{mkgrp}. Edits
will be performed on all the groups specified by the ID numbers in
\lstinline{list_of_group_ids}. At least one group id must be
specified.

\subsection{Removing groups}
To remove display groups, simply do:
\begin{lstlisting}
(ckg) rmgrp list_of_group_ids
\end{lstlisting}
If the active display group is removed, then group 0 becomes the new
active display group (unless there are no more groups in the project).

\section{Checkerboards}

Checkerboards are managed using similar commands to those used for
display groups. These commands only affect the current active display
group, so \texttt{chgrp} must be used before checkerboards in another
group can be managed.

As mentioned in the introduction, checkerboard patterns have a variety
of attributes. A full list of attributes follows:
\begin{compactemize}
\item dimensions (in number of unit cells)
\item size of the initial unit cell (in pixels)
\item size of the final unit cell (in pixels)
\item position (of the anchor, in pixels from bottom-right corner of window)
\item anchor (i.e. where the initial cell is positioned within the board
  e.g. top-left)
\item colors (which alternate both in space and time)
\item frequency of flashing / color alternation (in Hz)
\item initial phase of color alternation (in degrees)
\end{compactemize}

\subsection{Creating a checkerboard}
To create a checkerboard in the current display group, use
\texttt{mk}. The usage is as follows:
\begin{lstlisting}
mk dims init_unit end_unit position anchor cols freq [phase]
\end{lstlisting}
Except for the phase, which defaults to \texttt{0}, you have to
specify all the attributes mentioned above. For more detailed
information, enter \texttt{help mk}.

\subsection{Editing checkerboards}
Similar to \texttt{edgrp}, you can edit checkerboards in the current
group by using \texttt{ed} and specifying a list of checkerboard
ID numbers. The usage is as follows:
\begin{lstlisting}
ed [--dims WIDTH,HEIGHT] [--init_unit WIDTH,HEIGHT]
   [--end_unit WIDTH,HEIGHT] [--position X,Y]
   [--anchor LOCATION] [--cols COLOR1,COLOR2]
   [--freq FREQ] [--phase PHASE]
   list_of_checkerboard_ids
\end{lstlisting}
For more detailed information, enter \texttt{help ed}.

\subsection{Removing checkerboards}
Much like \texttt{rmgrp}, to remove checkerboards from the current
group, simply use \texttt{rm}:
\begin{lstlisting}
(ckg) rm list_of_checkerboard_ids
\end{lstlisting}

\section{Display}
To display the project on-screen, use the \texttt{display} command.
To stop displaying project animation midway, press ESC.  The usage is
as follows:
\begin{lstlisting}
display [-f] [-p LEVEL] [-r N] [-pt] [-lt] [-ld] [-ss] [-sp]
        [list_of_group_ids]
\end{lstlisting}
With no options specified, the project will be displayed in a window,
going through each display group in order. If you want display groups
to be displayed in a specific order, just supply a list of group ids
as arguments. If you want to repeat displaying groups in that order
several times, then give the \lstinline{-r/--repeat} option and
specify how many repeats you want. For example, entering:
\begin{lstlisting}
display -r 20 2 0 1 1
\end{lstlisting}
will result in group 2 being displayed first, then group 0, then group
1, then group 1 again, with the sequence of groups being shown 20
times in total.

The following subsections describe the other options in greater
detail.

\subsection{Fullscreen display}

The \lstinline{-f/--fullscreen} flag, when specified, causes the
project to be displayed in fullscreen. If the project's resolution is
not the same as the screen's resolution, then checkergen will try to
stretch the animation to fit the screen. This requires the use of
framebuffer objects, so your OpenGL installation must support the
\emph{GL\_EXT\_framebuffer\_object} extension.

\subsection{Setting the process priority}

To reduce the number of monitor refreshes that are missed, checkergen
can increase the priority of the Python process it is running in, but
thus far only on Windows. This functionality requires the pywin32
module, and is invoked by giving the \lstinline{-p/--priority} flag
and specifying the priority level.

There are 4 priority levels, low, normal, high and realtime, which can
be also specified by integers in the range 0 to 3 (i.e. 0 means low, 3
means realtime). When conducting an experiment, use 
\lstinline{display -p realtime} for the best performance. However, using
realtime priority might cause the computer to become less responsive
to user input in some cases.

\subsection{Logging time information}
Specifying the \lstinline{-lt/--logtime} flag will enable logging of
each frame's timestamp, while specifying the \lstinline{-ld/--logdur}
flag will enable logging of each frame's duration. This information
will be output to a log file in the same directory as the project file
once display is done, with the filename as
\texttt{projectname.log}. If either kind of logging is enabled, and if
signalling is also enabled (\lstinline{-sp} or \lstinline{-ss} flags),
then the log file will also record the trigger signals sent at each
frame, if there were any.

\subsection{Sending triggers}

To send trigger signals via the parallel port to the data acquisition
machine, specify the \lstinline{-sp/--sigpar} flag. To send signals
via the serial port, specify the \lstinline{-ss/--sigser} flag. Serial
port functionality has not, as of this date, been tested.

Unique triggers are sent upon several events:
\begin{compactemize}
\item When the checkerboards in a display group appear on-screen (42
  sent)
\item When the checkerboards in a display group disappear from the
  screen (17 sent)
\item Whenever a checkerboard undergoes a color reversal (64 + board
  id sent)
\end{compactemize}
Should two or more checkerboards happen to undergo a color reversal at
the same time, the trigger sent will be for the checkerboard
with the largest ID number.

Triggers are sent immediately after the screen flip operation returns,
ensuring as far as possible that the stimulus onset is synchronized
with the sending of the trigger. For a discussion of issues regarding
synchronization of the screen with the signal-sending port, see
section \ref{sec:phototest}.

\subsection{Photodiode testing}
\label{sec:phototest}

To facilitate testing of stimulus onset time and refresh rates using a
photodiode, the \lstinline{-pt/--phototest} flag can be
specified. When specified, a small white test rectangle will be
displayed (for the duration of one frame) in the top-left corner of the
window/screen whenever the checkerboards in a display group appear
on-screen. Hence, this should coincide with a trigger sent by the
parallel or serial port.

However tests have shown that on certain set-ups, the trigger is sent
by the port one frame earlier than when the photodiode signal is
detected, suggesting that the trigger is sent before the screen flip
operation has completed entirely. If this happens to you, it may be
due to vertical sync (vsync) not being properly enabled. To play safe,
it is advisable to configure OpenGL or your graphics driver to force
vsync on for all applications. After doing so, the port signal should
be synchronized with the photodiode signal.

\section{Export}
The \texttt{export} command is similar to the \texttt{display}
command, allowing you to specify the list of group ids to exported, as
well as allowing you to specify the length of the project animation
that should be exported (in seconds). Only PNG is supported as an
export format. The usage is as follows:
\begin{lstlisting}
usage: export [-n] [-r N] [duration] [dir] [list_of_group_ids]
\end{lstlisting}
For more detailed information, enter \lstinline{help export} into the
checkergen prompt.

\end{document}
